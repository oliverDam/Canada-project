%preamble - package inclusion and set up
\input{setup/preamble.tex}

%macros - please read this file
\input{setup/macros.tex}

\begin{document}       % TIP: If you are using TeXstudio you can open
%\tableofcontents      %      the file by Ctrl+LeftClick on setup/macros.tex
%\pagebreak             %      If the file doesn't exist, you will be asked
					   %      weather or not you want to create it.
%\begin{center}
%	\vspace{5cm}
%	\Huge{Worksheets}
%\end{center}
%\clearpage

%||||||||||||||||||||||||||||||||||||||||||||||||||||||||||||||||
%|||||||                 Example Inputs                  ||||||||
%||||||||||||||||||||||||||||||||||||||||||||||||||||||||||||||||
%|||||||                                                 ||||||||
%			 \input{chapters/aFigureSample.tex}			 %|||||||
%			 \input{chapters/bTableSample.tex} 		     %|||||||
%			 \input{chapters/cEquationSample.tex}		 %|||||||
%			 \input{chapters/dTikzSample.tex}            %|||||||
%			 \input{chapters/eCodeSample.tex}            %|||||||
%|||||||                                                 ||||||||
%||||||||||||||||||||||||||||||||||||||||||||||||||||||||||||||||
%||||||||||||||||||||||||||||||||||||||||||||||||||||||||||||||||


%%% Prereport %%%
		\setlength\cftaftertoctitleskip{2pt}
		\setlength\cftafterloftitleskip{6pt}
		\setlength\cftafterlottitleskip{6pt}
%\selectlanguage{danish}
%\title{Testing the performance of linear regressors using inertial information combined with sEMG to minimize the limb position effect in proportional and simultaneous control of lower arm prosthetics.}

%%% Frontmatter Settings %%%
		\pagestyle{empty} %disable headers and footers
		\pagenumbering{roman} %use roman page numbering in the frontmatter I II...
%		\fancyfoot[RE,LO]{17gr7404} %page number on all pages
%		\fancyfoot[LE,RO]{\thepage}
%		\fancyhead[LE,LO,RE,RO]{}

%%% Introductory Formalities %%%
%\includepdf[pages={1}]{formalities/frontpage.pdf}
			\clearpage
\thispagestyle{empty}

\begin{figure}[H]
	\raggedleft{
	\includegraphics[width=0.3\textwidth]{setup/aaulogo-en.png} }
	\hspace{4.5cm}
	\raggedright{
	\includegraphics[width=0.35\textwidth]{setup/RNL.png} }
\end{figure} 

\vspace{3 cm}

\begin{center}
	\begin{Huge}
	%	\textbf{- Status Seminar -}\\
	%	\vspace{2cm}
		\textbf{Developing a wearable system to assess balance during advanced dynamic movements}\\ 
	\end{Huge}
%\vspace{5mm}
%\begin{Large}
%	\textbf{- A pilot study -}\\
%\end{Large}
		\vspace{20 mm}
		\begin{Huge}
		3rd semester Masters, Biomedical Engineering \& Informatics - Autumn $2018$\\
		\vspace{3 mm}
	\end{Huge}
	{\Large Project group: $18$gr$9406$} \\
	\vspace{1cm}
	\large{Simon Bruun and Oliver Thomsen Damsgaard}
\end{center}
\vspace*{\fill}

\begin{center}
	\line(1,0){400}
\end{center}

%\newpage
%
%\large{\textbf{Project period:}\\
%P7, Autumn 2017\\
%01/08/2017 - 20/12/2017\\
%
%\textbf{Project group:}\\
%17gr7404\\} %\fxnote{Input group number}
%
%
%\begin{center}
%	\Large{\textbf{Collaborators:}\\
%		\vspace{1.5cm}
%	\rule{10cm}{1pt}\\
%	Irene Uriarte \\
%	
%	\rule{10cm}{1pt}\\
%	Martin Alexander Garenfeld \\
%	
%	\rule{10cm}{1pt}\\
%	Oliver Thomsen Damsgaard \\
%	
%	\rule{10cm}{1pt}\\
%	Simon Bruun \\}
%\end{center}
%
%
%
%\large{\textbf{Supervisors:}\\
%Strahinja Dosen \\
%Jakob Lund Dideriksen \\
%Lotte N.S. Andreasen Struijk} \\
%\\
\newpage
			% <--- the frontpage
%			\pagestyle{fancy}
%%\include{formalities/kolofon}
%			\input{formalities/titleSheet.tex} 			 % <--- the titlesheet - contains the synopsis!!
%%%% Preface %%%
%			\cleardoublepage
%			%\lipsum[15]
%write real thing
\textbf{\textit{Abstract -}} Current rehabilitation methods for stroke and spinal cord injury patients are inadequate in preparing patients for independent daily life. Additionally assessment methods are susceptible to errors as patients are qualitatively evaluated. The effect of this is observed in the fact that up to 39\% of stroke patients and 45\% of spinal cord injury patients experience falls post rehabilitation.Studies have suggested that alternative training methods, involving advanced dynamic movements such as Tai Chi and Pilates, could be an improvement to currently used methods. This study propose to develop a wearable system with a combination of force sensitive resistors and gyroscopes to evaluate on balance performance during advanced dynamic movements, such as karate, to test if it is possible to quantify performance and balance at different experience levels of karate. \textbf{\textit{Method}} Three subject were included, with varying experience at karate. Force sensitive resistors were placed under the soles to measure pressure distribution during a specific karate sequence. Gyroscopes were placed at the knees to measure angular velocities. Data were filtered with a third order low pass Butterworth filter, cutoff at 2.5Hz for pressure data, cutoff at 1.25Hz for gyroscope data. A performance score was calculated, based on four measures. Depending on data-distribution, a one-way ANOVA or Kruskal-Wallis statistical test was used. Bonferroni correction was applied to correct for false results. \textbf{\textit{Results}} No significant differences were found between measures. A significant difference in performance score was found between the novice and master subjects ($p<0.05$).\\ \textbf{\textit{Conclusion}} The study concludes that balance is improved with experience in karate. However, the authors recommend that further studies should be conducted in the research area in order to determine the possibility of quantifying advanced dynamic movements with wearable systems.			 % <--- this is the abstract!!
%%\clearpage
%			\input{formalities/forord.tex}				% <--- the preface
%

%\clearpage
%			\pdfbookmark[0]{Table of Contents}{label: tableOfCentents}
%			\tableofcontents
%			\cleardoublepage


%%% Mainmatter Settings %%%
\pagenumbering{arabic} %use arabic page numbering in the mainmatter
\fancyhf{}
\fancyfoot[C]{\thepage} %\text{ of} \pageref{LastPage}			% ADD LABLE{LASTPAGE} TO LAST PAGE !!
\fancyfoot[RE,LO]{18gr9406}																								   %
\fancyhead[RE,LO]{}																												%% } consider fancyfoots
\fancyhead[RE,LO]{\color{aaublue}\small\nouppercase\leftmark} %even page - chapter title %
\pagestyle{fancy}


%---------------------------INPUTS-------------------------------

%\input{contents/statusText.tex}
%\chapter{Introduction} \label{chap:Introduction}



\chapter{Status Seminar} \label{chap:PA}

\section{Spinal Cord Injury and Cardiovascular Diseases}

The spinal cord (SC) is part of the central nervous system (CNS) together with the brain. The SC is connecting the brain to the rest of the body by connecting to the peripheral nervous system (PNS). It is responsible for leading nerve impulses between the brain and body, to modulate movements, sensory inputs and visceral innervation. The SC extends from the brain just below the cranium down the spine to the lumbar vertebrae one and two (L1-L2). From L1-L2 to the end of the spine at the coccyx vertebrae or tailbone, bundles of nerve fibres extend further. The vertebrae bones encapsulates and protects the SC. However, trauma to the spine can cause trauma to the SC as well. \cite{Weidner2017}

Cardiovascular diseases (CVD) are the number one cause of death on a world wide scale. In 2015 CVDs was estimated to account for more than 31\% of all deaths globally. \cite{whocvd2017} CVD are a collected term for a number of conditions revolving around diseases to the heart and system of blood vessels. According to the World Health Organisation (WHO) the top two causes of global deaths are the CVDs coronary artery disease and stroke. Of the two, stroke accounted for 10\% of deaths in 2016. \cite{whoMortalityStats2018}


\section{Rehabilitation}

Patients suffering from neural damage caused by stroke or SCI will in many cases need an ambulatory assistive device to increase their independence when it comes to getting around in their every day life. \cite{Sandrini2018,Michael2005} %This need is caused by the previously mentioned effects both SCI and stroke can have on a patients ability to control their limbs and thereby also their balance.

\subsection{Spinal Cord Injury}

The aim of rehabilitating patients suffering from both stroke and SCI, is to compensate for the abilities they have lost by training the intact parts of their sensory-motor (SM) system. This can be achieved by activating the intact parts of the SM system with sensory cues recruiting both spinal and supra spinal connections. \cite{Sandrini2018}\\
This approach has been shown to work in animal studies, where neural systems in the spinal cord responsible for locomotion were trained independently of the connection to the brain. These methods have been used as the foundation for the current training protocols to train functional movements such as walking in patients with incomplete SCI, meaning the training should make use of the neural plasticity. \cite{Sandrini2018}\\
Training the walking ability includes a treadmill on which the patient will attempt to walk while being supported by an unloading harness. The treadmill walking should activate locomotion movements through the input from load and stretch sensitive mechanoreceptors, resulting in improved coordination, speed and strength. This training method can consist of both explicit and implicit methods. In the explicit method the patient will receive visual feedback to adjust the length of their steps in order to activate a cognitive process to adjust their gait. The implicit method will rely on resistance in order to train locomotion without the patient having to plan their step length. Another important aspect of rehabilitation of these individuals is their balance, and lately training of this ability has been shown to increase both speed and distance in walking tests. \cite{Sandrini2018}

\subsection{Stroke}

Gait and balance rehabilitation of stroke patients focuses on implementing both the locomotion mechanisms, but also the brain, to achieve functional gait on various surfaces. Several studies have indirectly shown that cortical functions are involved in the gait cycle, where they are responsible for reacting and adapting the gait to uneven surfaces. \cite{Belda2011} \\
As in cases of SCI, stroke rehabilitation also focuses on regaining the ability to walk, and therefore the rehabilitation process implements gait training. Stroke patients have the same possibility to exploit motor plasticity, in order to relearn previously known skills. \cite{Belda2011} \\
There are many approaches to gait rehabilitation, where the most relevant in this case is the motor learning techniques. This method gives the patient an active role in the rehabilitation process, and there are multiple approved approaches within this technique. The overall thought behind this technique is to learn new and improve current skills by repetition of specific tasks and utilizing sensory feedback to achieve manipulation of the involved neural systems. \cite{Belda2011}

\subsection{Balance and Gait Training}

Studies have shown that dual-task mobility training helps improve balance and gait compared to groups that performed single-task training in stroke patients. The dual-task approach was designed to make the patient walk on a treadmill while performing either a cognitive or motor task at the same time. \cite{He2018}\\
Another approach can be seen in studies where Tai Chi was used as a rehabilitation method for stroke and Parkinson's disease patients, implementing the aspect of thought and simultaneous movement into the training \cite{Ding2012,Winser2018}. This use of martial arts training resulted in multiple studies finding significantly higher improvement in balance compared to the control groups, while gait measures did not improve significantly with the implementation of Tai Chi training. \cite{Ding2012}
These findings can not lead to a final conclusion due to the number of trials and sample size, but the results indicate that martial arts could help increase balance in stroke patients \cite{Ding2012}. It was also found that Tai Chi helps to reduce the number of falls for people suffering from balance problems after both stroke and Parkinson's disease, while in this study it did not result in a significant difference between balance measures compared with regular treatment \cite{Winser2018}. It has also been found that Pilates training improves both static and dynamic balance in older adults compared to the control group that only did their normal daily activities \cite{Moreno2017}.

\section{Assessing gait in rehabilitation}

Several different methods for assessing patients gait abilities have been developed to evaluate on the rehabilitation progress. Many of these outcome measures are also used in studies on the development of new technologies and methods. According to the Italian Robotic Neurorehabilitation Research Group (IRNRG) there are six outcome measures which are used more commonly in research studies. These six are: Functional Ambulation Category (FAC), the 10-m Walking Test, the Motricity Index, the 6-Minute Walking Test, the Rivermead Mobility Index and the Berg Balance Scale (BBS). \cite{Sandrini2018}

\subsection{Measuring Gait In a Clinical Environment}

Recent technological improvements makes it possible to perform advanced gait analysis (GA) while examining 3D kinematics and EMG in a clinical setting. This method provides the clinician with an advanced insight in the patients current abilities and gives the possibility of measuring and quantizing any changes that might occur during a rehabilitation process. \cite{Sandrini2018}\\
The method of using 3D kinematics takes place in a laboratory with the use of cameras, surface electromyography (sEMG), force platforms and stereophotogrammetry equipment to provide the needed data to perform GA. The system provides recordings for qualitative analysis, as well as quantitative measures of muscle activation, contact forces with the ground and body position during gait. These measures are used to evaluate the gait cycle with regards to step length, cadence, swing time, rotation and power in the joints for the individual subject. \cite{Sandrini2018}\\
An attempt to quantify the quality of gait with a single parameter was made with the Normalcy Index (NI), where the algorithm measures deviation of a patients gait pattern from the gait of healthy individuals through Principal Component Analysis (PCA). The mean pattern is based on some of the features obtained with GA. This method has been proven to be an effective tool to examine changes in gait over time. \cite{Sandrini2018}
Further advances in the quantification of the many features is the gait deviation index, the gait profile score and the movement deviation profile. All of these methods take different approaches to finding the deviation between healthy gait and the measured variables from the advanced clinical set-up described briefly above. \cite{Sandrini2018}\\
Other approaches exist to measure improvement during rehabilitation. One study calculated the combined centre of pressure of the patient and a walking frame (WF) as a combined system, by measuring reaction forces of both the patient and the WF along with cameras capturing the placement of the feet relative to the WF. This gave the possibility to calculate the weight supported through the frame and the stability of both patient and WF. \cite{Costamagna2017}

\subsection{Measuring Gait Outside a Clinical Environment}

Methods of measuring gait and other dynamic activities outside clinical environments are becoming more accessible and favourable over measurement methods bound to clinical environments. Rehabilitation and assessment in clinical environments rarely translate well to real life situations \cite{Basteris2014}. Such systems are most functional if they are wearable by the patient or test subject. \\
Wearable systems to analyse and monitor body dynamics are attracting an increased interest in research, where accelerometers and inertial measurement units (IMU) are the most used newer studies. Here, studies have used wearable systems to measure upper limb kinematics and trunk posture, to evaluate on movement performance. \cite{Wang2017} Wearable systems can also be used for assessing gait by implementing multiple sensors placed on the subjects lower limbs, measuring variables such as acceleration, gyroscopic and pressure forces and EMG depending on what system is implemented. Here, measuring forces applied to the feet can be done with simple force sensors based on either resistive, piezoelectric or capacitive designs, and often includes an implementation of these in shoes or insoles. \cite{Muro2014} A study by Muro et al. \cite{Muro2014} has been shown that the implementation into insoles reflects the measurements obtained from clinical motion analysis laboratories. \\
Inertial measurement units (IMU) can also be implemented in wearable devices. These consist of gyroscopes measuring the rotational inertia used to measure changes in direction, as well as accelerometers measuring the acceleration in three axes giving the opportunity to measure changes in balance or sudden knocks such as those experienced by the sensor while walking. \cite{Muro2014}

\section{Future works}

The future works of the project will focus on measuring balance with a wearable system during advanced movements (Pinan Nidan kata sequence in this case). Our supervisor at UVic would like to quantify improvement of coordination and balance during advanced movements in relation to rehabilitation of balance after stroke or SCI. This leads to our current problem. 

\textit{Is it possible to develop a wearable system to measure balance and coordination during advanced movements in relation to rehabilitation?}

\subsection{Methods}

The system we are most likely to end up with will be a combination of IMU's placed on different parts of the body as well as sensors in the soles to measure centre of pressure during the sequence. The aim is to measure the rotational forces and use these to quantify the coordination of different limb movements, as well as using them to separate the different movements of the sequence. Pressure sensors in the soles will be used to quantify the body sway and thereby the balance in between these movements, where the subjects should be standing in a stable position with weight on both feet. 

We are most likely going to use FlexiForce sensors for the soles along with an amplifier and data logger. The IMU's will be Consensys Shimmers as these are easy to set up and test on different sites of the body. As the movements within the sequence should start at the head going down through the body, we are expecting to place sensors at the top of the neck, on the back, at the hips and on the calves. 

To evaluate the system we will perform tests on both trained and untrained subjects, as well as measuring the change in coordination and balance in untrained subjects over multiple days of training. In the future this could be used by our supervisor to analyse the improvement of balance during advanced, non-locomotor-movements in stroke and SCI patients, as he wants to examine if focusing more on advanced movements during rehabilitation could lead to higher functionality and balance. 


%\chapter{Methods} \label{chap:Methods}


%\chapter{Results} \label{chap:Results}


%\chapter{Discussion} \label{chap:Discussion}


%\chapter{Conclusion} \label{chap:Conclusion}




\clearpage
%\begin{multicols}{2}
	
\urlstyle{same}
\printbibliography			% to bibliography: "compile" -> Tools -> Commands -> "Bibtex" -> "compile/build"
\clearpage
%\end{multicols}


%\cleardoublepage
% BILAG
%\begin{appendices}
%	\chapter{Appendices}
%
%\end{appendices}


\end{document}
