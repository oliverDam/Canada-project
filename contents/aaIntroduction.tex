%introduction

This is introduction. 

Stroke and spinal cord injuries are bad for people. Not in relation to causing deaths, because mentioning this in the introduction would imply that we tend to fix that problem. We do not. 

Stroke and SCI are bad for people who like to walk. To have normal independent daily life. Then, people are send through rehabilitation. 
However, current rehabilitation are not very good at getting people good at going again. This is seen as a lot of people experience falls post rehabilitation. This is bad. Real bad.

What is the cause? We are not sure, but a plausible cause could be that current rehabilitation methods are not properly training people to be fit for independent daily life. Current rehabilitation methods for training people involves walking on treadmills. Walking is a fine way of training people to walk. However, it is bad at training them to walk outside clinical environments, out in real life. This has been proven. (add some references)
Additionally, current rehabilitation methods for assessing peoples health and ability to walk properly could be flawed in that it is qualitative, meaning that it is subjected to changes from session to session, patients immediate feeling and energy level and the physicians personal experience. 

Thus, it would be real nice to have a new way to train patients to gain good ability to walk. This would also involve them having good balance, strength and coordination of movements. 
Some kind of dynamic movement might be better training. Luckily such training exist. Martial arts involves dynamic movement. Some martial arts even involves dynamic movements on an advanced level. Karate is one such martial art. 
Additionally, is would be really cool to have some way to assess how well people are at performing karate, to investigate if they get better balance, strength and coordination over time. 

Thus, this project will investigate the possibility to develop a system to assess the performance of the martial arts of karate (and maybe suggest that rehabilitation should begin to incorporate new methods for training people, and not just keep doing what it has done for the last 20 years.)

