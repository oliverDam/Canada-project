%unused from aStroke.tex


\subsection{Current stroke rehabilitation (and its shortcomings)}
Consequently, many stroke patients are at a higher risk of falling. %, much higher than the risk normally associated with ageing. 
The prevention of patients falling should be prioritized as falls can lead to loss of independence and serious injury. Despite the consequences, studies have shown that 30-39\% of stroke patients fall at least once during a rehabilitative process, and of these patients 42\% experience multiple falls. \cite{Bhalla2016, Hanger2014} Apart from the immediate risks and dangers to patients falling, falls can also further extend the rehabilitation period and worsen the rehabilitation of both motor and cognitive functions \cite{Wong2016, Blennerhassett2012}. Current rehabilitation and ambulatory does not facilitate high activity levels for patients. A study by Kathleen et al. /cite{simons kilde 8} have shown patients to have extremely low levels of daily activity (avg. 2837 steps/day) when compared to the norm (5000-6000 steps/day) for sedentary elderly people (65-70 years). Without adequate training and activity patients have little chance of regaining lost mobility, which are as previously described can lead to loss of confidence in performing movements, loss of independence, prolonged rehabilitation period and an increased risk of falls and injury. 





%discusses too much on possible new rehabilitation methods and does not lead to why we should develop a new evaluation device/method
Current rehabilitation often involves gait treadmill training for rehabilitating dynamic movements /cite{simons kilde 8}}. However, as often is with training of patients in clinical environments, this approach translates poorly to real life outside the clinical environment \cite{Basteris2014}. In daily life sudden changes in walking speed or change of direction is normal and necessary, however this is often not accounted for on a treadmill in a clinical environment. To improve muscle strength of patients, weights or cables have been used to apply resistance to leg movement when walking on a treadmill, however this approach is cumbersome and cannot be used outside clinical environments. 
Different methods have been researched to explore ways to both train balance and strength in patients gait. A study by Washabaugh et al. \cite{Washabaugh2016}, developed a device which could be worn outside clinical environments to put resistance on leg movements by applying resistant force at the knee joint. The study showed a significant increase in muscle activity during gait with applied resistance \cite{Washabaugh2016}. Other studies have used assistive robotic devices in gait training. Chisari et al. \cite{Chisari2015} used an assistive robotic device to show significant improvements in hemiparetic patients walking ability, but not in leg muscle strength. Motor imaging training have also been shown to significantly improve on patients gait and leg muscle strength \cite{Kuma2015}. 



%something on that most studies only focus/use treadmill trianing and dont do much in strength training eventhough this is very important. -> this can lead to why we like to do multimodal training which can incorporate both balance AND strength training. 

%kilde 1: Individualized Treadmill and Strength Training for Chronic Stroke Rehabilitation: Effects of Imbalance (søgt adgang hos AUB)
		%study investigate effect of treadmill-strength training. 

%more on the (hopefully) proven/significant worsening of patient performance in balance/gait/dynamic movement following falls compared to non-fallers. [THIS STUDY IS NOT ON FALLS]

%people get stroke -> stroke lead to bad balance and movement -> people fall -> falls are bad -> current rehab does not make patients not fall -> new rehab -> need new evaluation method -> our project

%it would be nice to put patients through a rehabilitation program which effectively increase their balance and performance of dynamic movements. therefore it is necessary to be able to evaluate if patients get better balance/movement





