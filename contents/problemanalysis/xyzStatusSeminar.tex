\section{Future works}

The future works of the project will focus on measuring balance with a wearable system during advanced movements (Karate kata Pinan Nidan sequence in this case). Our supervisor at UVic would like to quantify improvement of coordination and balance during advanced movements in relation to rehabilitation of balance after stroke or SCI. This leads to our current problem. 
\begin{center}
\textit{Is it possible to measure and quantify balance and coordination of movements in rehabilitation focusing on dynamic movements?}
\end{center}
\subsection{Methods}

The system we are most likely to end up with will be a combination of IMU's placed on different parts of the body as well as sensors in the soles to measure centre of pressure during the sequence. The aim is to measure the rotational forces and use these to quantify the coordination of different limb movements, as well as using them to separate the different movements of the sequence. Pressure sensors in the soles will be used to quantify the body sway and thereby the balance in between these movements, where the subjects should be standing in a stable position with weight on both feet. 

We are most likely going to use FlexiForce sensors for the soles along with an amplifier and data logger. The IMU's will be Consensys Shimmers as these are easy to set up and test on different sites of the body. As the movements within the sequence should start at the head going down through the body, we are expecting to place sensors at the top of the neck, on the back, at the hips and on the calves. 

To evaluate the system we will perform tests on both trained and untrained subjects, as well as measuring the change in coordination and balance in untrained subjects over multiple days of training. In the future this could be used by our supervisor to analyse the improvement of balance during advanced, non-locomotor-movements in stroke and SCI patients, as he wants to examine if focusing more on advanced movements during rehabilitation could lead to higher functionality and balance. 