\section{Experiment protocol} %protokol er ikke det rigtige ord
For this project reaction and pressure forces is collected from subjects performing the karate kata Pinan Nidan. %Reaction forces are measured from the subjects body rotations and movements, while pressure forces are measured from under the subjects feet. 
A karate kata is a sequence of detailed patterns of movements. Many different types of kata exist, each practice visualisation, balance and basic technique through repetition of movements. Different katas have different sequences of movements, some are more difficult than others where jumps and kicks are part of the movements. For this projects data acquisition the kata Pinan Nidan is chosen. Pinan Nidan consists of a series of movements involving steps, turning and hand strikes, where the performers' feet are on the ground at most times. It is the first kata of the Wado-Ryu system and is taught to new students as it is seen as an easier kata for beginners. \cite{Mccarthy1987, Dojo2018} Therefore it should be possible to quickly teach a complete newcomer to karate through Pinan Nidan and have an adequate performance of the kata to compare to the movements of a master of the kata. 

During performance of the Pinan Nidan kata, subjects will be wearing three Shimmer3 devices, located on the head, chest and waist [MAYBE]. Subjects will also be wearing shoes with FSR installed and a belt mounted Arduino Uno for data collection from the FSRs. 

% consider where this text fits in
The FSRs will be installed in a pair of shoes for subjects to use during data acquisition, to ensure the same placement of sensors. One FSR 406 is placed under the heel and two FSR 402 sensors are placed under the front of the foot at the left and right side of the anterior lateral eminence of the sole. One Arduino Uno will be placed at the back of the hip of the subject to handle data collection. Collected data will be stored on an SD card for later analysis with MATLAB. 



