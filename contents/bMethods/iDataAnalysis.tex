\section{Data Analysis}
This section covers the analysis of acquired data.

\subsection{Calculation of Movement Scores}

For calculation of a movement score four separate measures were calculated; length, span, velocity change and movement intensity. Center of Pressure (COP) was calculated for a test subject on a plane and used as base for the calculation of length and span measures. The length was calculated as the path of the moving COP during recording. Span was calculated as the area in which the COP has travelled. Velocity change was used as a measure for the frequency of the changes in angular velocity. Movement intensity was used to describe how fast a subject moved during the kata performance. 

\subsubsection{Calculation of Centre of Pressure}
%method for calculation of COP 
%MATLAB COP code description
The COP calculation consisted of two simple equations to find the displacement of balance in the X and Y directions. To ensure the calculated values were unrelated to the weight of the test subject, but solely describes the displacement of balance, each calculation of the pressure distribution will be divided by the overall pressure placed on all sensors at that point. The COP\lowercase{x} equation found the distribution of weight between the two feet, whereas the COP\lowercase{y} equation describes the distribution between the sensors on the front and back of the foot. The COP measure was used in calculations of the length and span measures. A weight ($W$) was added to the pressure readings ($P$) to compensate  for the displacement of the sensors in relation to each other and for the number of sensors on the front and back of the foot.
The COP calculations for X and Y directions followed \eqref{eq:COPx} and \eqref{eq:COPy}:

\begin{equation} \label{eq:COPx}
COP_x(i) =  \frac{\sum_{i=1}^{3}P_i W_i - \sum_{i=4}^{6}P_i W_i}{\sum_{i=1}^{6}P_i W_i}
\end{equation} 

\begin{equation} \label{eq:COPy}
COP_y(i) =  \frac{\sum(P_3 W_3 + P_6 W_6) - \sum(P_1 W_1+P_2 W_2+P_4 W_4+P_5 W_5)}{\sum_{i=1}^{6}P_i W_i}
\end{equation} 


%method for calculating scores based on calculated COP and gyroscope data

\subsubsection{Calculation of Length score}
The length of the COP outcomes was calculated and divided by the length ($L$) of the recorded data, so the outcome measure described the mean COP change between each sample. To ensure this measure had an effect on the final score it was multiplied by a factor of 10. The length was calculated individually for X and Y directions for later use in score calculation (see \eqref{eq:length}).

\begin{equation} \label{eq:length}
Length_{x,y} = \frac{\sum_{i=1}^{L-1}\sqrt{(COP_{x,y} (i+1)-COP_{x,y} (i))^2}}{L} * 10
\end{equation}


\subsubsection{Calculation of Span score}
Calculation of the span described the span between the outer most points of the COP changes, giving the area in which the COP travelled. 
The span was calculated by taking the absolute value of the difference between maximum and minimum observed values of COP. This was calculated for both the X and Y directions. In the same manner as the length measure scaled by a factor of 10, the span measure was divided by 10 to decrease the effect of the span in relation to the other measures. Span was calculated as shown in \eqref{eq:span}:
%Start out by finding min and max of scaled COP and difference between these values. Sum and abs of difference to find final span value. This is then divided by 10 for scaling in score calculation.

\begin{equation} \label{eq:span}
Span_{x,y} = \frac{\left| max(COP_{x,y})-min(COP_{x,y})\right|}{10}
\end{equation}


\subsubsection{Calculation of Velocity Change}
%As described in \subref{subsec:filtering} on filtering, it is shown that the power spectrum of the gyroscope data provided very little information above 10Hz. Thus, the frequencies have been divided into two categories: low and high. Here, low frequencies is defined as 2/3 of the frequency spectrum from 0 to 5Hz, and are an expression for stable movements. High frequencies are defined as the last 1/3 of the frequency spectrum from 0 to 5Hz, and are an expression for less stable movements. This separation of low and high frequencies, relative to the frequency band of acquired data, is based on when the frequency distribution deviated less than 5\% of the mean for individual subject. The frequency distribution is calculated by the difference between the power at frequencies in the low and high category, divided by the total power of frequencies between 0 and 5Hz, as shown in \eqref{eq:fd}:

The calculation of velocity changes processed the angular velocities as frequencies to determine the changes in angular velocities. A Fourier transform of the data showed the distribution of slow and fast changes of the angular velocity. Here, the divide between slow changes (frequencies: 0.01-0.625Hz) and fast changes (frequencies: 0.626-1.25Hz) are set at half the cutoff frequency used for filtering (1.25Hz).
The fast changes were assumed to be caused by trembles at the legs done to correct and regain balance. Slow changes were hypothesised to only be caused by larger movements, such as steps or turns. Thus, the measure is an expression for how precise movements were performed. The measure was calculated by subtracting frequencies in the low category from the high category frequencies, and dividing by to sum of all frequencies, as shown in \eqref{eq:VC}:
%low 0.01 - 0.625
%high 0.626 - 1.25


\begin{equation} \label{eq:VC}
Change = \frac{LowFreq-HighFreq}{LowFreq+HighFreq}
\end{equation}


\subsubsection{Calculation of Movement Intensity}
The movement intensity measure express how fast a subject could start and stop movements. The calculation divided the distribution of slow and fast movements, based on the measured angular velocity. The border between slow and fast was decided as the mean angular velocity measured for a full recording. This means that the faster a subject moved during the kata the higher the mean would be. This ensured that only very fast movements such as steps and turns were categorised as fast velocities, while slower movements such as those made during sways when regaining balance were categorized as slow velocities. The categorization of movements were calculated as shown in \eqref{eq:MI}, where $i$ is sample and $j$ is the channel for each of the three gyroscope channels (x, y, z). Following, the calculation for movement intensity is as shown in \eqref{eq:MI}, where $L$ is the size of the gyroscope data matrix.

\begin{equation} \label{eq:category}
M(i,j) = \left \{
\begin{matrix}
{mean(Gyro(j)) < Gyro(i,j) = 1}\\{mean(Gyro(j)) > Gyro(i,j) = 0}
\end{matrix} \right.
\end{equation}



%\begin{equation} \label{eq:category}
%M(i,j) = \{
%\stackrel{mean(Gyro(j)) < Gyro(i,j) = 1}{mean(Gyro(j)) > Gyro(i,j) = 0}
%\end{equation}
%
%mean(Gyro(j)) < Gyro(i,j) = 1
%mean(Gyro(j)) > Gyro(i,j) = 0


\begin{equation} \label{eq:MI}
Intensity = \frac{M*\left| Gyro\right| }{\sum_{i=1}^{L}\left| Gyro\right| }
\end{equation}


\subsubsection{Calculation of Performance score}
Each movement score; length, span and velocity change and movement intensity was used to calculate a final performance score for each subject. 
The performance score was calculated as seen in \eqref{eq:score}. The best performance score was produced by having a larger span relative to the length, as the performance score is better the lower the value. %this is stupid yes but we wont change it 

%Having a short length for a movement sequence would mean precise path directly from start to finish, while a longer length would mean that a longer path have been taken between start and finish. A great span measure is a result of greater force excessed on the FSRs, meaning that movements have been performed faster. 
%lille lenght = præcis movement

\begin{equation} \label{eq:score}
Score = \frac{1-Length}{Span} * (1 - Change)*(1-Intensity)
\end{equation}

