%the shimmer device
\section{Data Acquisition}

For this project data will be acquired using accelerometers and force sensors. The accelerometer is provided through the use of the Shimmer3 device from ShimmerSensing. Force sensors are from Interlink Electronics of the 400 Series. %Data will be collected using MATLAB. 

The Shimmer3 device is a nine degree of freedom (DoF) Inertial Measurement Unit (IMU) possessing five different types of sensors; accelerometer, gyroscope, magnetometer and altimeter. The Shimmer3 is capable of being configured to enable or disable specific sensors depending on which is needed. For this project only the accelerometer module of the device will be used. The accelerometer in the Shimmer3 device is a LSM303DLHC. It has a three dimensional digital linear acceleration sensor with a range of $\pm$2g / $\pm$4g / $\pm$8g / $\pm$16g. \cite{LSM303DLHC} %It has 16bit data output and communicates through an I$^{2}$C serial interface. 
Communication between the Shimmer3 devices and the computer is through Bluetooth. The computer will be running MATLAB and the \textit{Shimmer MATLAB Instrument Driver Library} to collect the streamed data from the Shimmer3 device. 

The force sensors used in this project are Force Sensing Resistors (FSR) from Interlink Electronics, models 402 and 406. The FSR 402 is a 13mm diameter circle single-zone resistor capable of force detection in a range from 20g to 2kg. The FSR 406 is similar but covers a larger area of 38mm in a square. \cite{IE400}
A total of six sensors will be used with three sensors under each foot. The FSRs will be installed in a pair of shoes for subjects to use during data acquisition, to ensure the same placement of sensors. One FSR 406 is placed under the heel and two FSR 402 sensors are placed under the front of the foot at the left and right side of the anterior lateral eminence of the sole. One Arduino Nano will be placed at the back of the hip of the subject to handle data collection. Collected data will be stored on an SD card for later analysis with MATLAB. 
%[foot regions: https://www.sciencephoto.com/media/581111/view/anatomy-regions-of-the-right-foot]
%(https://www.digikey.ca/product-detail/en/interlink-electronics/30-73258/1027-1002-ND/2476470) big square sensor
%(https://www.digikey.ca/product-detail/en/interlink-electronics/30-81794/1027-1001-ND/2476468) small round sensor


For this project reaction and pressure forces is collected from subjects performing the karate kata Pinan Nidan. %Reaction forces are measured from the subjects body rotations and movements, while pressure forces are measured from under the subjects feet. 
A karate kata is a sequence of detailed patterns of movements. Many different types of kata exist, each practice visualisation, balance and basic technique through repetition of movements. Different katas have different sequences of movements, some are more difficult than others where jumps and kicks are part of the movements. For this projects data acquisition the kata Pinan Nidan is chosen. Pinan Nidan consists of a series of movements involving steps, turning and hand strikes, where the performers feet are on the ground at most times. It is the first kata of the Wado-Ryu system and is taught to new students as it is seen as an easier kata for beginners. \cite{Mccarthy1987} Therefore it should be possible to quickly teach a complete newcomer to karate through Pinan Nidan and have an adequate performance of the kata to compare to the movements of a master of the kata. 

During performance of the Pinan Nidan kata, subjects will be wearing three Shimmer3 devices, located on the head, chest and waist. Subjects will also be wearing shoes with FSR installed and a belt mounted Arduino Nano for data collection from the FSRs. 
