%\section{Experiment Protocol / description} %protokol er ikke det rigtige ord

%this will act as the initil part of the methods section. as a header for the chapter

For this project reaction and pressure forces are collected from subjects performing the karate kata Pinan Nidan. %Reaction forces are measured from the subjects body rotations and movements, while pressure forces are measured from under the subjects feet. 
A karate kata is a sequence of detailed choreographed patterns of movements. Many different types of kata exist, each practice visualisation, balance and basic technique through repetition of movements. Different katas have different sequences of movements, some are more difficult than others where jumps and kicks are part of the movements. For this projects data acquisition the kata Pinan Nidan is chosen. 
Pinan Nidan consists of a series of movements involving steps, turning and hand strikes, where the performers' feet are on the ground at most times. Pinan Nidan takes between 30 and 60 seconds to perform depending on the speed of movements. The kata consists of 13 stepping, 11 turning, 7 punching and 13 blocking movements. \cite{Mccarthy1987, seikenryu2017, Dojo2018}

\subsection{Subjects} %put here instead of having it as its own section as it is very short
For this project two healthy test subjects will be used; one who is a master at Karate, and one who is novice. Both subjects are able-bodied and have no neurological or muscular injuries. Both subjects were prior to the test instructed about the purpose of the study and their role as test subjects. Both the master and novice were instructed to perform the karate kata Pinan Nidan as best they are able to. The novice further received detailed instruction on how to perform the kata, as the novice has never before done karate. 


%During performance of the Pinan Nidan kata, subjects will be wearing three Shimmer3 devices, located on the head, chest and waist [MAYBE]. Subjects will also be wearing shoes with FSR installed and a belt mounted Arduino Uno for data collection from the FSRs. 




