%\section{Experiment Protocol / description} %protokol er ikke det rigtige ord

%this will act as the initil part of the methods section. as a header for the chapter

For this project reaction and pressure forces were collected from subjects performing the karate kata Pinan Nidan. %Reaction forces are measured from the subjects body rotations and movements, while pressure forces are measured from under the subjects feet. 
A karate kata is a sequence of detailed choreographed patterns of movements. Many different types of kata exist, each practice visualisation, balance and basic technique through repetition of movements. Different katas have different sequences of movements, some are more difficult than others where jumps and kicks are part of the movements. For this projects data acquisition the kata Pinan Nidan is chosen. 
Pinan Nidan consists of a series of movements involving steps, turning and hand strikes, where the performers' feet are on the ground at most times. Pinan Nidan takes between 30 and 60 seconds to perform depending on the speed of movements. The kata consists of 13 stepping, 11 turning, 7 punching and 13 blocking movements \cite{seikenryu2017}.

\subsection{Subjects} %put here instead of having it as its own section as it is very short
Three healthy test subjects were included in this study; one who is a master at karate (+30 years of karate experience), one intermediate (3-5 years karate experience) and one novice (less than 1 year of karate experience). All subjects were able-bodied and had no neurological or muscular injuries. Subjects were prior to the test instructed about the purpose of the study and their role as test subjects.


%During performance of the Pinan Nidan kata, subjects will be wearing three Shimmer3 devices, located on the head, chest and waist [MAYBE]. Subjects will also be wearing shoes with FSR installed and a belt mounted Arduino Uno for data collection from the FSRs. 


