%discussion
\section{Discussion}

number of test subjects are few. Not too much of a problem since we only aim to prove that a system such as this would be able to work.

FSR#3 broke. This does of course make flaws in our data acquisition, since the FSRs were all old and used prior to our study. This is a problem according to Hall et al. \cite{Hall2008}. But we do not use the FSRs for very long and, as state in the test section, FSR data is not used to compare between FSR outputs, but only used for calculating COP for subjects. Additionally, the "error" in FSR#3 is consistent for all data acquired and should therefore not have an effect in our study. This does however mean that our study cannot be used to compare to other studies.

use gyroscopes to determine rotation speed of movements and rotations

inclusion of accelerometers to measure movement acceleration and velocities not in rotational axes. 





further/future discussion points and/or what should be studied next:

a more in-depth report should be done on the effects of martial arts training in rehabilitation. This could be tested with our system, a similar or more developed one. Additionally, comparison should be made between rehabilitation evaluations done with a system like ours and the older currently used methods for assessing progress in rehab. a study like that would be a step to determining if rehab training should be changed and as well if assessing methods for rehab training should be changed.

combine this easily wearable system with kinetic/kinematics movement analysis (marker analysis of human movement) to have a better way of evaluating this type of system than qualitatively looking at measurements and compare with regular video footage. 