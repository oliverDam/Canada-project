%discussion
%\section{Discussion}

The expected results was that the master subject with the most experience in karate would perform the best, followed by the intermediate with second best results and novice performing the worst. This was reflected by the scores, where the master ($7.278 \pm 0.891$) had a significantly higher score ($p<0.05$) than the novice ($9.532 \pm 0.292$), while there was no difference between the intermediate and the novice or the master. This was caused by a combination of the different measures, where the most influencing measure was the intensity. This outcome measure yielded a significant difference ($p<0.05$) between the novice ($0.758 \pm 0.013$) and the master ($0.783 \pm 0.003$), and was the main reason for the different score outcomes, along with small variations in the other measures. The reason for this could be that the novice subject sways more when it stops a movement and has to regain balance, while performing slower and less intense movements as well, leading to a lower intensity score. This does not necessarily indicate the level of skill when performing the kata, but it shows the novice was not as stable and does not perform as explosive movements during the kata compared to the master.

During the test of the system a problem occurred with one of the FSRs used for measuring pressure distribution during the kata performance. The test of the FSRs showed that FSR3 returned lower values than the other sensors. This was most likely caused by the fact that the FSRs used in the project were used in another study as well. This problem has been highlighted by Hall et al. \cite{Hall2008}. As the FSR data was not used for comparison between FSRs and the deviation in measured resistance was consistent for all measurements, it was assumed that it did not have an effect on the final outcome. However, a consequence of this is that the measures in this study cannot be compared to other studies implementing the same methods, as the FSR readings would differ from this study. However the relation between the measures is still relevant for comparison.

A shortcoming of this study is that few subjects were recruited. This provided less evidence for the study results, but the study was meant as an initial process in the research area of investigating new methods for development of wearable systems to use in rehabilitation programs utilizing advanced dynamic movements in place of traditional treadmill gait training. This study aimed to prove the concept of developing a simple wearable system, in comparison to more advanced and costly techniques such as kinematic motion analysis. 

The authors suggest that future studies should investigate the methods presented in this study in relation to kinematic motion analysis outcomes. This could be done in order to quantitatively evaluate the method of this study, when no similar studies, utilizing FSRs and gyroscopes, have yet been conducted. 

In addition to this study, future studies could include the use of accelerometers to further expand on ways to analyse motion. Accelerometers could be used to investigate movement in translational axes, and not only rotational axes as in this study, as well as give a more precise indication for when and how movement occurs. Additionally, more sensors could be included to acquire more data. Other studies have used several more sensors under the feet \cite{Hessert2005, Hu2018}.

Future studies should further investigate the use of martial arts as training in rehabilitation. Few studies have investigated the effect of training with martial arts or similar training including more advanced dynamic movements training \cite{Winser2018, Ding2012}, however these studies have suggested improvement in patients strength and balance. It should be investigated further if this type of training alone or combined with traditional treadmill gait training currently used in rehabilitation programs could improve the outcome of the rehabilitation.




%number of test subjects are few. Not too much of a problem since we only aim to prove that a system such as this would be able to work.

%FSR\#3 broke. This does of course make flaws in our data acquisition, since the FSRs were all old and used prior to our study. This is a problem according to Hall et al. \cite{Hall2008}. But we do not use the FSRs for very long and, as state in the test section, FSR data is not used to compare between FSR outputs, but only used for calculating COP for subjects. Additionally, the "error" in FSR\#3 is consistent for all data acquired and should therefore not have an effect in our study. A consequence of this is, however, that this study cannot be used to compare to other studies, while other studies would have fully functional FSRs. 

%use gyroscopes to determine rotation speed of movements and rotations

%inclusion of accelerometers to measure movement acceleration and velocities not in rotational axes.

%discussion on result results





%further/future discussion points and/or what should be studied next:

%a more in-depth report should be done on the effects of martial arts training in rehabilitation. This could be tested with our system, a similar or more developed one. Additionally, comparison should be made between rehabilitation evaluations done with a system like ours and the older currently used methods for assessing progress in rehab. a study like that would be a step to determining if rehab training should be changed and as well if assessing methods for rehab training should be changed.

%combine this easily wearable system with kinetic/kinematics movement analysis (marker analysis of human movement) to have a better way of evaluating this type of system than qualitatively looking at measurements and compare with regular video footage. 