%conclussion

%short presentation of the aim of the project, followed by the findings. 
%Stroke and SCI patients are not too good after rehab. Karate might make them better. Our findings say that karate might be able to improve on stroke or SCI patients, as our results show that balance, accuracy and speed of movement is improved with experience in karate. 

%nothing can be concluded from this study as it has included too few subjects to really prove anything. It can however pave the way for the development of newer method to use in rehabilitation of 

%This study proposed a novel method of measuring balance during advanced dynamic movements, and found a significant difference between test subjects with different skill levels within the performed kata. The results suggest that there is a relation between the length and the span of the Center of Pressure, where a high length within a low span is a sign of bad balance, while a low length within a high span shows the subject has control of balance. At the same time high frequencies could indicate small corrections of posture, while low frequencies represent precise and controlled movements. 

This study proposed a novel method of measuring balance during advanced dynamic movements, and found a significant difference between test subjects with different skill levels within the performed kata. The results suggest that there is a relation between the length and the span of the Center of Pressure, where a high length within a low span is a sign of bad balance, while a low length within a high span shows the subject has control of balance. At the same time, the frequency distribution and intensity of gyroscopic data was found to be related to the intensity and precision of the performed movements.