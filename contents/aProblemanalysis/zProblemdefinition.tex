%problemformuleringsafsnit
\section{Problem definition}

The previous chapter introduces the problem of patients suffering from complication of mobility caused by stroke or SCI and leads to the current rehabilitating methods used to train patients to regain mobility. 
%% this section should be made clear in badnessOfCurrentRehab.tex
However, these methods are not standardised and evaluations of patient performance is subjected to physicians personal experience and patients immediate feeling. Additionally, most evaluations occur in clinical environments when performing simple movement tasks, which translate poorly to real life. 
%%-------------------

It can be discussed whether or not current rehabilitation training with single-task training of simple movements is properly prepares patients to live independent daily lives post rehabilitation. This project propose that training involving advanced dynamic movements can further improve on patients strength and balance, better preparing them for daily life, when compared to traditional simple movement training. 
However, there exist no suitable system to evaluate advanced dynamic movements. Thus a new system is needed to have a way to measure and assess patients ability to perform advanced dynamic movements to determine if this type of training is an improvement to current rehabilitation methods.

This leads to the following problem definition:

%Thus a new method or system is needed to have a standardised and easily assessed method to frequently measure patient improvement for advanced movements in mobility rehabilitation.

\begin{center}
How can a wearable system, capable of measuring balance performance during advanced dynamic movements, be developed.
\end{center}
