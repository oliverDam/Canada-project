%problemformuleringsafsnit
\section{Problem definition}

Following the previous chapter the work leads us from the problem of patients suffering from complication of mobility caused by stroke or SCI, to the current rehabilitating methods used to train patients to regain mobility. However, these methods are not standardised and evaluations of patient performance is subjected to physicians personal experience and patients immediate feeling. Additionally, most evaluations occur in clinical environments when performing simple movement tasks, which translate poorly to real life. Thus a new method or system is needed to have a standardised and easily assessed method to frequently measure patient improvement for advanced movements in mobility rehabilitation.

\begin{center}
How is it possible to develop a wearable system to measure performance of advanced dynamic movements to evaluate a subjects performance in relation to balance, coordination and sequence of execution.
\end{center}
