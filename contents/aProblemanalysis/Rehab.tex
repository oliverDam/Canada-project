%this is rehab
\section{Rehabilitation}

Patients suffering from neural damage caused by stroke or SCI will in many cases need to go through a rehabilitation process to regain or relearn lost functions \cite{Sandrini2018,Michael2005}. Currently many different methods for rehabilitation exist, many with focus on patients regaining the ability to control their limbs and balance, as a step toward regaining independence. An important aspect of rehabilitation is that when assigning training to patients of SCI or stroke, it is important to evaluate the state of the patient, as different patients will have different levels of dysfunction depending on the severity of the damage caused by SCI or stroke. Rehabilitation should be suited to each patient individually. \cite{Sandrini2018}

\subsection{Current stroke rehabilitation methods}

Gait and balance rehabilitation of stroke patients focus on implementing the locomotion mechanisms, but also the brain, to achieve functional gait on various surfaces. Several studies have indirectly shown that cortical functions are involved in the gait cycle, where they are responsible for reacting and adapting the gait to uneven surfaces. \cite{Belda2011} \\
As in cases of SCI, stroke rehabilitation also focuses on regaining the ability to walk, and therefore the rehabilitation process implements gait training. Stroke patients have the same possibility to exploit motor plasticity, in order to relearn previously known skills. \cite{Belda2011} \\
There are many approaches to gait rehabilitation, where the most relevant in this case is the motor learning techniques. This method gives the patient an active role in the rehabilitation process, and there are multiple approved approaches within this technique. The overall thought behind this technique is to learn new and improve current skills by repetition of specific tasks and utilizing sensory feedback to achieve manipulation of the involved neural systems. \cite{Belda2011}

\subsection{Current SCI rehabilitation methods}

The aim of rehabilitating patients suffering from either stroke or SCI, is to compensate for the abilities they have lost by training the intact parts of their sensory-motor (SM) system. This can be achieved by activating the intact parts of the SM system with sensory cues recruiting both spinal and supra spinal connections. \cite{Sandrini2018}\\
This approach has been shown to work in animal studies, where neural systems in the spinal cord responsible for locomotion were trained independently of the connection to the brain. These methods have been used as the foundation for the current training protocols to train functional movements such as walking in patients with incomplete SCI, meaning the training should make use of the neural plasticity. \cite{Sandrini2018}\\
Training the walking ability includes a treadmill on which the patient will attempt to walk while being supported by an unloading harness. The treadmill walking should activate locomotion movements through the input from load and stretch sensitive mechanoreceptors, resulting in improved coordination, speed and strength. This training method can consist of both explicit and implicit methods, where patients will either receive visual feedback to adjust the length of their steps in order to activate a cognitive process to adjust their gait. The implicit method will rely on resistance in order to train locomotion without the patient having to plan their step length. Another important aspect of rehabilitation of these individuals is their balance, and lately training of this ability has been shown to increase both speed and distance in walking tests. \cite{Sandrini2018}

% Balance and gait training have been moved to own .tex file


%2. Assesment of gait disorders in neurorehabilitation page 69:
%https://link-springer-com.zorac.aub.aau.dk/content/pdf/10.1007%2F978-3-319-72736-3.pdf 
%
%Something about rehabilitation, balance confidence and stability w. multimodal self administered balance training: 3. https://www.ncbi.nlm.nih.gov/pmc/articles/PMC5119910/ 
%and with strength training: 4. https://www.ncbi.nlm.nih.gov/pmc/articles/PMC3885846/ 
%
%Something about the balance over the course of acute rehabilitation: 5. https://www.archives-pmr.org/article/S0003-9993(95)81035-8/abstract
%
%7. Accelerometer mounted on people: http://iopscience.iop.org/article/10.1088/0967-3334/37/10/1785
%
%8. Something with peoples activity when they have had a stroke: https://www.sciencedirect.com/science/article/pii/S0003999305001905
%
%9. Stroke rehab guidelines: http://journals.sagepub.com/doi/pdf/10.1177/1747493016643553
%
%10. SCI guidelines side 227, measuring side 238, appropriate outcome measures side 76, stroke 187 https://link-springer-com.zorac.aub.aau.dk/content/pdf/10.1007%2F978-3-319-72736-3.pdf
%
%11. Stroke Rehab:
%https://jneuroengrehab.biomedcentral.com/articles/10.1186/1743-0003-8-66
%
%12. New methods for gait analysis page 235:
%https://link-springer-com.zorac.aub.aau.dk/content/pdf/10.1007%2F978-3-319-72736-3.pdf
%
%13. Something with Tai Chi:
%https://journals.lww.com/ajpmr/FullText/2012/12000/Tai_Chi_for_Stroke_Rehabilitation__A_Focused.10.aspx 
%
%14. Something with dual-task vs. single-task:
%http://journals.sagepub.com/doi/full/10.1177/0269215518758482
%
%15. More on tai chi:
%http://journals.sagepub.com/doi/full/10.1177/0269215518773442
%
%16. Pilates:
%https://journals.humankinetics.com/doi/full/10.1123/japa.2017-0078
%
%What we want is to measure reliance on walking device plus activity patterns.