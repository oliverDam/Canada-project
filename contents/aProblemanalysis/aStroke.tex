%stroke

\section{Cardiovascular Diseases}

%Short introduction to cvd, stroke and the two subtypes
Cardiovascular diseases (CVD) are the number one cause of death on a world wide scale. In 2015 CVDs was estimated to account for more than 31\% of all deaths globally. \cite{whocvd2017} CVD are a collected term for a number of conditions revolving around diseases to the heart and system of blood vessels. According to the World Health Organisation (WHO) the top two causes of global deaths are the CVDs coronary artery disease and stroke. Of the two, stroke accounted for 10\% of deaths in 2016. \cite{whoMortalityStats2018}
%something on why we focus on stroke rather than coronary artery disease...?

\subsection{Stroke}
%stroke in numbers and causes /cites: (Hering2016chap7 and 8), Mackay2002, Sharma2017
A stroke is caused by either a blockage or rupture of blood vessels in the brain. As such stroke is divided into two subtypes; ischaemic stroke and haemorrhagic stroke. During an ischaemic stroke a blood vessel in the brain is blocked by blood cloths caught in narrow blood vessels. The narrowing of blood vessels are commonly caused by other conditions such as high cholesterol, high blood pressure, unhealthy lifestyle and ageing. If a blood vessel is blocked a part of the brain will be shut off from its blood supply. If not treated within minutes this can cause damage to brain cells in the cut-off area. \cite{Mackay2002, Hering2016chap7, InternetStroke2018} 
During an haemorrhagic stroke a blood vessel will rupture and blood will leak inside the brain. Depending on where in the brain the leak occurs the haemorrhagic stroke is either a intracerebral haemorrhage or and subarachnoid haemorrhage, intracerebral being inside the brain and subarachnoid occurring in the space between the brain and the cranium. In both types a rupture and leakage of blood can cause a sudden increase in pressure potentially causing damage to brain cells and can lead to sudden unconsciousness and death. The most common causes are high blood pressure, unhealthy lifestyle, diabetes and ageing. \cite{Mackay2002, Hering2016chap8, InternetStroke2018}

\subsection{Stroke Complications}
%post stroke consequences \cite{Sharma2017}
%outcomes/consequences, damages to balance/coordination of movements /cites: Bhalla2016, Sharma2017
Complications following a stroke are common. In surviving patients 30-96\% have been reported to experience post-stroke complications of both physical and psychological nature. Complications involve recurrent stroke and epileptic seizures, cardiac arrhythmias and failure, infections, problems in gastrointestinal and genitourinary systems, complication of immobility, dementia, pain and depression.  \cite{Bhalla2016} %[short version:] Consequences include numerous complications of recurrent stroke, mobility, pain and depression.
Thus, the consequences are many, however this study will focus on complications of mobility. Following a stroke complications related to movement are common. Depending on where in the brain the stroke occurs it can have a variety of outcomes that can affect the patients balance and motor control \cite{Zehr2011}. Up to 38\% of stroke patients have been reported to experience spasticity affecting the performance of dynamic muscle movements such as gait. Spasticity and motor control changes can occur following an upper motor neuron lesion. \cite{Bhalla2016} %, where part of the brain responsible for motor control or neural pathways above the anterior horn cell of the spinal cord have been damaged. \cite{Bhalla2016} 
Stroke patients are also at a higher risk of osteoporosis due to weakening of performing voluntary movements or movements as a whole if the patient experience hemiparesis \cite{Bhalla2016,Zehr2011}.
