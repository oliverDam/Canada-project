\subsection{Balance and Gait Training}
%this section could go before badnessOfCurrentRehab.tex so that badnessOfCurrentRehab is right before the Problemdifinition.tex

% more focus on dual task training to lead to training with advanced dynamic movements like karate/tai chi
In contrast to current rehabilitation methods, mainly focusing on simple gait training, a rhythmic movement, newer approaches have begun to use dual-task training, incorporating cognitive tasks as well. Similarly, training involving advanced movements have suggested to improve balance for both stroke and Parkinson's disease patients \cite{Ding2012,Winser2018}. 

Studies have shown that dual-task mobility training helps improve balance and gait compared to groups that performed single-task training in stroke patients. The dual-task approach was designed to make the patient walk on a treadmill while performing either a cognitive or motor task at the same time. \cite{He2018}
The walking/motor dual-task method proved to be significantly better at improving speed, stride length and cadence for both dual-task and single-task tests. Combining walking and cognitive tasks improved the patients cadence and dynamic gait index, which describes balance while walking, in single-task tests. It was also found that combining balance and cognitive or motor tasks improved a number of balance measures significantly compared to single-task training. \cite{He2018}
Despite the outcomes reported in \cite{He2018}, the conclusion is that more studies are needed in order to support that dual-task training improves performance in dual-task tests. The review study shows that a dual-task approach improves single-task tests compared to the single-task training. \cite{He2018}

A similar approach can be seen in studies where Tai Chi was used as a rehabilitation method for stroke and Parkinson's disease patients, implementing the aspect of thought and simultaneous movement into the training \cite{Ding2012,Winser2018}. This use of martial arts training resulted in multiple studies finding significantly higher improvement in balance compared to the control groups, while gait measures did not improve significantly with the implementation of Tai Chi training. \cite{Ding2012}
These findings can not lead to a final conclusion due to the number of trials and sample size, but the results indicate that martial arts could help increase balance in stroke patients \cite{Ding2012}. It was also found that Tai Chi helps to reduce the number of falls for people suffering from balance problems after both stroke and Parkinson's disease, while in this study it did not result in a significant difference between balance measures compared with regular treatment \cite{Winser2018}. It has also been found that Pilates training improves both static and dynamic balance in older adults compared to the control group that only did their normal daily activities \cite{Moreno2017}.