%section on outcome measures used to assess patients physical functionality and the rehabilitation progress.
\section{Assessing gait in rehabilitation}

Several different methods for assessing patients gait abilities have been developed to evaluate on the rehabilitation progress. Many of these outcome measures are also used in studies on the development of new technologies and methods. According to the Italian Robotic Neurorehabilitation Research Group (IRNRG) there are six outcome measures which are used more commonly in research studies. These six are: Functional Ambulation Category (FAC), the 10-m Walking Test, the Motricity Index, the 6-Minute Walking Test, the Rivermead Mobility Index and the Berg Balance Scale (BBS). \cite{Sandrini2018}

\subsection{Measuring Gait In a Clinical Environment}

Recent technological improvements makes it possible to perform advanced gait analysis while examining 3D kinematics and EMG in a clinical setting. This method provides the clinician with an advanced insight in the patients current abilities and gives the possibility of measuring and quantizing any changes that might occur during a rehabilitation process. \cite{Sandrini2018}

The method of using 3D kinematics takes place in a laboratory with the use of cameras, sEMG, force platforms and stereophotogrammetry equipment to provide the needed data to perform gait analysis. The system provides recordings for qualitative analysis, as well as quantitative measures of muscle activation, contact forces with the ground and body position during gait. These measures are used to evaluate the gait cycle with regards to step length, cadence, swing time, rotation and power in the joints for the individual subject. \cite{Sandrini2018}

An attempt to quantify the quality of gait with a single parameter was made with the Normalcy Index (NI), where the algorithm measures deviation of a patients gait pattern from the gait of healthy individuals through Principal Component Analysis (PCA). The mean pattern is based on some of the features obtained with GA (\textbf{what is GA}). This method has been proven to be an effective tool to examine changes in gait over time. \cite{Sandrini2018}

Further advances in the quantification of the many features is the gait deviation index, the gait profile score and the movement deviation profile. All of these methods take different approaches to finding the deviation between healthy gait and the measured variables from the advanced clinical set-up described briefly above. \cite{Sandrini2018}

\subsection{Measuring Gait Outside a Clinical Environment}

There are many other approaches to the measurement of improvement during rehabilitation. One study calculated the combined centre of pressure of the patient and a walking frame (WF) as a combined system, by measuring reaction forces of both the patient and the WF along with cameras capturing the placement of the feet relative to the WF. This gave the possibility to calculate the weight supported through the frame and the stability of both patient and WF. \cite{Costamanga2017}

Another study examined the importance of accelerometer position, age and walking speed on the accuracy of accelerometer based measurement of gait. It was found that the device location did not affect measures such as speed, cadence and single limb support time. Gait asymmetry and variability was shown to be affected by the previously mentioned factors. (\textbf{what are these factors? er det alder og walking speed nævnt i første sætning, eller de measures som nævnes i forrige sætning?}) \cite{Hurwitz2016}

%This study aims to measure improvement of reliance on a WF outside the clinical setting, as well as change in activity in everyday life based on a simple system with inspiration from what was developed in \cite{Costamanga2017} and \cite{Hurwitz2016}. The outcome measures that should be achieved with the system is stability, reliance on WF based on the weight distribution between the patient and the WF, number of steps taken as well as an overview of the daily activity of the patients before, during and after rehabilitation.
% det er ikke længere det vi gør

\subsection{Wearable Systems Used in Gait Assessment}

Wearable systems for gait assessment implements multiple sensors placed on the subject, measuring variables such as acceleration, gyroscopic and pressure forces and EMG depending on what system is implemented. \cite{Muro2014}

Measuring forces applied to the feet can be done with simple force sensors based on either resistive, piezoelectric or capacitive designs, and often includes an implementation of these in shoes or insoles. A study by Muro et al. \cite{Muro2014} has been shown that the implementation into insoles reflects the measurements obtained from clinical motion analysis laboratories. 

Inertial measurement units (IMU) can also be implemented in wearable devices. These consist of gyroscopes measuring the rotational inertia used to measure changes in direction, as well as accelerometers measuring the acceleration in three axes giving the opportunity to measure changes in balance or sudden knocks such as those experienced by the sensor while walking. \cite{Muro2014}



%The FAC is a method of categorizing patients into six levels depending on the patients ability to walk unaided. The categories range from the patient being able to walk anywhere unaided to the patient not being able to walk at all. Steps of gradually needing more support to be able to walk exist between to the extremes. \cite{Sandrini2018}  

%The 10 meter walking test (10mWT) is a simple test of measuring the time it takes a subject to walk 10 meters. Either steps can be counted or time can be measured
%%no need to describe the tests. 




