%section on outcome measures used to assess patients physical functionality and the rehabilitation progress.
\section{Assessing gait in rehabilitation}

Several different methods for assessing patients gait abilities have been developed to evaluate on the rehabilitation progress. Many of these outcome measures are also used in studies on the development of new technologies and methods. According to the Italian Robotic Neurorehabilitation Research Group (IRNRG) there are six outcome measures which are used more commonly in research studies. These six are: Functional Ambulation Category (FAC), the 10-m Walking Test, the Motricity Index, the 6-Minute Walking Test, the Rivermead Mobility Index and the Berg Balance Scale (BBS). \cite{Sandrini2018}

%The FAC is a method of categorizing patients into six levels depending on the patients ability to walk unaided. The categories range from the patient being able to walk anywhere unaided to the patient not being able to walk at all. Steps of gradually needing more support to be able to walk exist between to the extremes. \cite{Sandrini2018}  

%The 10 meter walking test (10mWT) is a simple test of measuring the time it takes a subject to walk 10 meters. Either steps can be counted or time can be measured
%%no need to describe the tests. 




