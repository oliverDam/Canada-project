%this is rehab
\section{Rehabilitation}

A lot of the patients with stroke or SCI will need an ambulatory assistive device to increase their independence when it comes to moving around in their every day life. (2, 8) This need is caused by the previously mentioned effects both SCI and stroke can have on a patients ability to control their balance. (kilde fra olivers afsnit). 

\subsection{Spinal Cord Injury}

The aim of rehabilitating patients suffering from SCI is to compensate for the abilities they have lost by training the intact parts of their sensory-motor (SM) system. This can be achieved by activating the intact parts of the SM system with sensory cues recruiting both spinal and supra spinal connections. (10)

This approach has been shown to work in animal studies, where neural systems in the spinal cord responsible for locomotion were trained independently of the connection to the brain. These methods have been used as the foundation for the current training protocols to train functional movements such as walking in patients with incomplete SCI, meaning the training should make use of the neural plasticity. (10)

Training the walking ability includes a treadmill on which the patient will attempt to walk while being supported by an unloading harness. The treadmill walking should activate locomotion movements through the input from load and stretch sensitive mechanoreceptors, resulting in improved coordination, speed and strength. (10)

Write about explicit and implicit learning.


1. Load cells implemented in shoes: https://www.sciencedirect.com/science/article/pii/S1350453317301510

2. Many are able to walk for a short distance: https://www.ncbi.nlm.nih.gov/pmc/articles/PMC4066430/ 

Something about rehabilitation, balance confidence and stability w. multimodal self administered balance training: 3. https://www.ncbi.nlm.nih.gov/pmc/articles/PMC5119910/ and with strength training: 4. https://www.ncbi.nlm.nih.gov/pmc/articles/PMC3885846/ 

Something about the balance over the course of acute rehabilitation: 5. https://www.archives-pmr.org/article/S0003-9993(95)81035-8/abstract and with VR: 6. https://www.liebertpub.com/doi/abs/10.1089/cpb.2005.8.187

7. Accelerometer mounted on people: http://iopscience.iop.org/article/10.1088/0967-3334/37/10/1785

8. Something with peoples activity when they have had a stroke: https://www.sciencedirect.com/science/article/pii/S0003999305001905

9. Stroke rehab guidelines: http://journals.sagepub.com/doi/pdf/10.1177/1747493016643553

10. SCI guidelines: https://link-springer-com.zorac.aub.aau.dk/content/pdf/10.1007%2F978-3-319-72736-3.pdf

What we want is to measure reliance on walking device plus activity patterns.