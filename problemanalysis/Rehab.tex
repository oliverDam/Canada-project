%this is rehab
\section{Rehabilitation}

Patients suffering from neural damage caused by stroke or SCI will in many cases need an ambulatory assistive device to increase their independence when it comes to getting around in their every day life. (2, 8) This need is caused by the previously mentioned effects both SCI and stroke can have on a patients ability to control their limbs and thereby also their balance. (kilde fra olivers afsnit). 

\subsection{Spinal Cord Injury}

The aim of rehabilitating patients suffering from both stroke and SCI, is to compensate for the abilities they have lost by training the intact parts of their sensory-motor (SM) system. This can be achieved by activating the intact parts of the SM system with sensory cues recruiting both spinal and supra spinal connections. (10)

This approach has been shown to work in animal studies, where neural systems in the spinal cord responsible for locomotion were trained independently of the connection to the brain. These methods have been used as the foundation for the current training protocols to train functional movements such as walking in patients with incomplete SCI, meaning the training should make use of the neural plasticity. (10)

Training the walking ability includes a treadmill on which the patient will attempt to walk while being supported by an unloading harness. The treadmill walking should activate locomotion movements through the input from load and stretch sensitive mechanoreceptors, resulting in improved coordination, speed and strength. This training method can consist of both explicit and implicit methods, where patients will either receive visual feedback to adjust the length of their steps in order to activate a cognitive process to adjust their gait. The implicit method will rely on resistance in order to train locomotion without the patient having to plan their step length. Another important aspect of rehabilitation of these individuals is their balance, and lately training of this ability has been shown to increase both speed and distance in walking tests. (10)

\subsection{Stroke}

Gait and balance rehabilitation of stroke patients also focuses on implementing both the locomotion mechanisms, but also the brain, to achieve functional gait on various surfaces. Several studies have indirectly shown that cortical functions are involved in the gait cycle, where they are responsible for reacting and adapting the gait to uneven surfaces. (11)

As in cases of SCI, stroke rehabilitation also focuses on regaining the ability to walk, and therefore the rehabilitation process implements gait training. Stroke patients have the same possibility to exploit motor plasticity, in order to relearn previously known skills. (11)

There are many approaches to gait rehabilitation, where the most relevant in this case is the motor learning techniques. This method gives the patient an active role in the rehabilitation process, and there are multiple approved approaches within this technique. The overall thought behind this technique is to learn new and improve current skills by repetition of specific tasks and utilizing sensory feedback to achieve manipulation of the involved neural systems. (11)


1. Load cells implemented in shoes: https://www.sciencedirect.com/science/article/pii/S1350453317301510

2. Many are able to walk for a short distance: https://www.ncbi.nlm.nih.gov/pmc/articles/PMC4066430/ 

Something about rehabilitation, balance confidence and stability w. multimodal self administered balance training: 3. https://www.ncbi.nlm.nih.gov/pmc/articles/PMC5119910/ and with strength training: 4. https://www.ncbi.nlm.nih.gov/pmc/articles/PMC3885846/ 

Something about the balance over the course of acute rehabilitation: 5. https://www.archives-pmr.org/article/S0003-9993(95)81035-8/abstract and with VR: 6. https://www.liebertpub.com/doi/abs/10.1089/cpb.2005.8.187

7. Accelerometer mounted on people: http://iopscience.iop.org/article/10.1088/0967-3334/37/10/1785

8. Something with peoples activity when they have had a stroke: https://www.sciencedirect.com/science/article/pii/S0003999305001905

9. Stroke rehab guidelines: http://journals.sagepub.com/doi/pdf/10.1177/1747493016643553

10. SCI guidelines side 227, measuring side 238, appropriate outcome measures side 76, stroke 187 https://link-springer-com.zorac.aub.aau.dk/content/pdf/10.1007%2F978-3-319-72736-3.pdf

11. Stroke Rehab:
https://jneuroengrehab.biomedcentral.com/articles/10.1186/1743-0003-8-66

What we want is to measure reliance on walking device plus activity patterns.