%this is rehab
\section{Rehabilitation}

Patients suffering from neural damage caused by stroke or SCI will in many cases need an ambulatory assistive device to increase their independence when it comes to getting around in their every day life. (2, 8) This need is caused by the previously mentioned effects both SCI and stroke can have on a patients ability to control their limbs and thereby also their balance. (kilde fra olivers afsnit). 

\subsection{Spinal Cord Injury}

The aim of rehabilitating patients suffering from both stroke and SCI, is to compensate for the abilities they have lost by training the intact parts of their sensory-motor (SM) system. This can be achieved by activating the intact parts of the SM system with sensory cues recruiting both spinal and supra spinal connections. (10)

This approach has been shown to work in animal studies, where neural systems in the spinal cord responsible for locomotion were trained independently of the connection to the brain. These methods have been used as the foundation for the current training protocols to train functional movements such as walking in patients with incomplete SCI, meaning the training should make use of the neural plasticity. (10)

Training the walking ability includes a treadmill on which the patient will attempt to walk while being supported by an unloading harness. The treadmill walking should activate locomotion movements through the input from load and stretch sensitive mechanoreceptors, resulting in improved coordination, speed and strength. This training method can consist of both explicit and implicit methods, where patients will either receive visual feedback to adjust the length of their steps in order to activate a cognitive process to adjust their gait. The implicit method will rely on resistance in order to train locomotion without the patient having to plan their step length. Another important aspect of rehabilitation of these individuals is their balance, and lately training of this ability has been shown to increase both speed and distance in walking tests. (10)

\subsection{Stroke}

Gait and balance rehabilitation of stroke patients also focuses on implementing both the locomotion mechanisms, but also the brain, to achieve functional gait on various surfaces. Several studies have indirectly shown that cortical functions are involved in the gait cycle, where they are responsible for reacting and adapting the gait to uneven surfaces. (11)

As in cases of SCI, stroke rehabilitation also focuses on regaining the ability to walk, and therefore the rehabilitation process implements gait training. Stroke patients have the same possibility to exploit motor plasticity, in order to relearn previously known skills. (11)

There are many approaches to gait rehabilitation, where the most relevant in this case is the motor learning techniques. This method gives the patient an active role in the rehabilitation process, and there are multiple approved approaches within this technique. The overall thought behind this technique is to learn new and improve current skills by repetition of specific tasks and utilizing sensory feedback to achieve manipulation of the involved neural systems. (11)

\subsection{Balance and Gait Training}

Studies have shown that dual-task mobility training helps improve balance and gait compared to groups that performed single-task training in stroke patients. The dual-task approach was designed to make the patient walk on a treadmill while performing either a cognitive or motor task at the same time. (14)

The walking/motor dual-task method proved to be significantly better at improving speed, stride length and cadence for both dual-task and single-task tests. Combining walking and cognitive tasks improved the patients cadence and dynamic gait index, which describes balance while walking, in single-task tests. It was also found that combining balance and cognitive or motor tasks improved a number of balance measures significantly compared to single-task training. (14)

Despite the outcomes reported in (14), the conclusion is that more studies are needed in order to support that dual-task training improves performance in dual-task tests. The review study shows that a dual-task approach improves single-task tests compared to the single-task training. (14)

A similar approach can be seen in studies where tai chi was used as a rehabilitation method for stroke and Parkinson's disease patients, implementing the aspect of thought and simultaneous movement into the training (13,15). This use of martial arts training resulted in multiple studies finding significantly higher improvement in balance compared to the control groups, while gait measures did not improve significantly with the implementation of tai chi training. (13)

These findings can not lead to a final conclusion due to the number of trials and sample size, but the results indicate that martial arts could help increase balance in stroke patients (13). It was also found that Tai Chi helps to reduce the number of falls for people suffering from balance problems after both stroke and Parkinson's disease, while in this study it did not result in a significant difference between balance measures compared with regular treatment (15). It has also been found that Pilates training improves both static and dynamic balance in older adults compared to the control group that only did their normal daily activities (16).

\subsection{Measuring Gait In a Clinical Environment}

Recent technological improvements makes it possible to perform advanced gait analysis while examining 3D kinematics and EMG in a clinical setting. This method provides the clinician with an advanced insight in the patients current abilities and gives the possibility of measuring and quantizing any changes that might occur during a rehabilitation process. (12)

This method takes place in a lab with the use of cameras, sEMG, force platforms and stereophotogrammetry equipment to provide the needed data to perform gait analysis. The system will provide recordings for qualitative analysis, as well as quantitative measures of muscle activation, contact forces with the ground and body position during gait. These measures are used to evaluate the gait cycle with regards to step length, cadence, swing time, rotation and power in the joints for the individual subject. (12)

An attempt to quantify the quality of gait with a single parameter was made with the Normalcy Index (NI), where the algorithm measures deviation of a patients gait pattern from the gait of healthy individuals through PCA. The mean pattern is based on some of the features obtained with GA. This method has been proven to be an effective tool to examine changes in gait over time. (12)

Further advances in the quantification of the many features is the gait deviation index, the gait profile score and the movement deviation profile. All of these methods take different approaches to finding the deviation between healthy gait and the measured variables from the advanced clinical set-up described briefly above. (12)

\subsection{Measuring Gait Outside a Clinical Environment}

There are many other approaches to the measurement of improvement during rehabilitation. One study calculated the combined centre of pressure of the patient and a walking frame (WF) as a combined system, by measuring reaction forces of both the patient and the WF along with cameras capturing the placement of the feet relative to the WF. This gave the possibility to calculate the weight supported through the frame and the stability of both patient and WF. (1)

Another study examined the importance of accelerometer position, age and walking speed on the accuracy of accelerometer based measurement of gait. It was found that the device location did not affect measures such as speed, cadence and single limb support time. Gait asymmetry and variability was shown to be affected by the previously mentioned factors. (7)

This study aims to measure improvement of reliance on a WF outside the clinical setting, as well as change in activity in everyday life based on a simple system with inspiration from what was developed in (1) and (7). The outcome measures that should be achieved with the system is stability, reliance on WF based on the weight distribution between the patient and the WF, number of steps taken as well as an overview of the daily activity of the patients before, during and after rehabilitation.


1. Load cells implemented in shoes: https://www.sciencedirect.com/science/article/pii/S1350453317301510

2. Assesment of gait disorders in neurorehabilitation page 69:
https://link-springer-com.zorac.aub.aau.dk/content/pdf/10.1007%2F978-3-319-72736-3.pdf 

Something about rehabilitation, balance confidence and stability w. multimodal self administered balance training: 3. https://www.ncbi.nlm.nih.gov/pmc/articles/PMC5119910/ 
and with strength training: 4. https://www.ncbi.nlm.nih.gov/pmc/articles/PMC3885846/ 

Something about the balance over the course of acute rehabilitation: 5. https://www.archives-pmr.org/article/S0003-9993(95)81035-8/abstract

7. Accelerometer mounted on people: http://iopscience.iop.org/article/10.1088/0967-3334/37/10/1785

8. Something with peoples activity when they have had a stroke: https://www.sciencedirect.com/science/article/pii/S0003999305001905

9. Stroke rehab guidelines: http://journals.sagepub.com/doi/pdf/10.1177/1747493016643553

10. SCI guidelines side 227, measuring side 238, appropriate outcome measures side 76, stroke 187 https://link-springer-com.zorac.aub.aau.dk/content/pdf/10.1007%2F978-3-319-72736-3.pdf

11. Stroke Rehab:
https://jneuroengrehab.biomedcentral.com/articles/10.1186/1743-0003-8-66

12. New methods for gait analysis page 235:
https://link-springer-com.zorac.aub.aau.dk/content/pdf/10.1007%2F978-3-319-72736-3.pdf

13. Something with Tai Chi:
https://journals.lww.com/ajpmr/FullText/2012/12000/Tai_Chi_for_Stroke_Rehabilitation__A_Focused.10.aspx 

14. Something with dual-task vs. single-task:
http://journals.sagepub.com/doi/full/10.1177/0269215518758482

15. More on tai chi:
http://journals.sagepub.com/doi/full/10.1177/0269215518773442

16. Pilates:
https://journals.humankinetics.com/doi/full/10.1123/japa.2017-0078

What we want is to measure reliance on walking device plus activity patterns.