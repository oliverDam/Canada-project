%spinal cord injury

\section{Spinal Cord Injury}

The spinal cord (SC) is part of the central nervous system (CNS) together with the brain. The SC is connecting the brain to the rest of the body by connecting to the peripheral nervous system (PNS). It is responsible for leading nerve impulses between the brain and body, to modulate movements, sensory inputs and visceral innervation. The SC extends from the brain just below the cranium down the spine to the lumbar vertebrae one and two (L1-L2). From L1-L2 to the end of the spine at the coccyx vertebrae or tailbone, bundles of nerve fibres extend further. The vertebrae bones encapsulates and protects the SC. However, trauma to the spine can cause trauma to the SC as well. \cite{Weidner2017}

The incidence for spinal cord injuries (SCI) ranges from 15 to 39 million a year in industrialized countries. Most traumatic causes are a result of traffic accidents, falls and violence. Causes for non-traumatic SCI are degenerative diseases and tumours. In prevalence of traumatic SCI, men outnumber women at a ratio of 3:1, while the prevalence is near equal in non-traumatic SCI. Any injury to the SC causing neurological damage can lead to serious dysfunction depending on where the injury happens. This can lead to loss of sensory sensation and motor control and dysfunction to bladder, bowel and cardiovascular functions. 

loss of motor function lead to bad balance and poor performance of movements. this is bad for patients



Any damage to the SC can have 


Together with the brain, the spinal cord (SC) is the central nervous system (CNS), connecting to the rest of the body through the peripheral nervous system (PNS). 

The Spinal Cord (SC) is responsible for 



