%\lipsum[15]
%write real thing
\textbf{\textit{Abstract -}} Current rehabilitation methods for stroke and spinal cord injury patients are deficient in preparing patients for independent daily life. Additionally assessment methods are susceptible to errors as patients are qualitatively evaluated. The effect of this is observed in the fact that up to 39\% of stroke patients and 45\% of spinal cord injury patients experience falls post rehabilitation.Studies have suggested that alternative training methods, involving advanced dynamic movements such as Tai Chi and Pilates, could be an improvement to currently used methods. This study propose to develop a wearable system with a combination of force sensitive resistors and gyroscopes to evaluate on balance performance during advanced dynamic movements, such as karate, to test if it is possible to quantify performance and balance at different experience levels of karate. \textbf{\textit{Method}} Three subject were included, with varying experience at karate. Force sensitive resistors were placed under the soles to measure pressure distribution during a specific karate sequence. Gyroscopes were placed at the knees to measure angular velocities. Data were filtered with a third order low pass Butterworth filter, cutoff at 2.5Hz for pressure data, cutoff at 1.25Hz for gyroscope data. A performance score was calculated, based on four measures. Depending on data-distribution, a one-way ANOVA or Kruskal-Wallis statistical test was used. Bonferroni correction was applied to correct for false results. \textbf{\textit{Results}} No significant differences were found between measures. A significant difference in performance score was found between the novice and master subjects ($p<0.05$).\\ \textbf{\textit{Conclusion}} The study concludes that balance is improved with experience in karate. However, the authors recommend that further studies should be conducted in the research area in order to determine the possibility of quantifying advanced dynamic movements with wearable systems.